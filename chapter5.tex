\documentclass[]{article}
\usepackage{lmodern}
\usepackage{amssymb,amsmath}
\usepackage{ifxetex,ifluatex}
\usepackage{fixltx2e} % provides \textsubscript
\ifnum 0\ifxetex 1\fi\ifluatex 1\fi=0 % if pdftex
  \usepackage[T1]{fontenc}
  \usepackage[utf8]{inputenc}
\else % if luatex or xelatex
  \ifxetex
    \usepackage{mathspec}
  \else
    \usepackage{fontspec}
  \fi
  \defaultfontfeatures{Ligatures=TeX,Scale=MatchLowercase}
\fi
% use upquote if available, for straight quotes in verbatim environments
\IfFileExists{upquote.sty}{\usepackage{upquote}}{}
% use microtype if available
\IfFileExists{microtype.sty}{%
\usepackage{microtype}
\UseMicrotypeSet[protrusion]{basicmath} % disable protrusion for tt fonts
}{}
\usepackage[margin=1in]{geometry}
\usepackage{hyperref}
\hypersetup{unicode=true,
            pdfborder={0 0 0},
            breaklinks=true}
\urlstyle{same}  % don't use monospace font for urls
\usepackage{color}
\usepackage{fancyvrb}
\newcommand{\VerbBar}{|}
\newcommand{\VERB}{\Verb[commandchars=\\\{\}]}
\DefineVerbatimEnvironment{Highlighting}{Verbatim}{commandchars=\\\{\}}
% Add ',fontsize=\small' for more characters per line
\usepackage{framed}
\definecolor{shadecolor}{RGB}{248,248,248}
\newenvironment{Shaded}{\begin{snugshade}}{\end{snugshade}}
\newcommand{\AlertTok}[1]{\textcolor[rgb]{0.94,0.16,0.16}{#1}}
\newcommand{\AnnotationTok}[1]{\textcolor[rgb]{0.56,0.35,0.01}{\textbf{\textit{#1}}}}
\newcommand{\AttributeTok}[1]{\textcolor[rgb]{0.77,0.63,0.00}{#1}}
\newcommand{\BaseNTok}[1]{\textcolor[rgb]{0.00,0.00,0.81}{#1}}
\newcommand{\BuiltInTok}[1]{#1}
\newcommand{\CharTok}[1]{\textcolor[rgb]{0.31,0.60,0.02}{#1}}
\newcommand{\CommentTok}[1]{\textcolor[rgb]{0.56,0.35,0.01}{\textit{#1}}}
\newcommand{\CommentVarTok}[1]{\textcolor[rgb]{0.56,0.35,0.01}{\textbf{\textit{#1}}}}
\newcommand{\ConstantTok}[1]{\textcolor[rgb]{0.00,0.00,0.00}{#1}}
\newcommand{\ControlFlowTok}[1]{\textcolor[rgb]{0.13,0.29,0.53}{\textbf{#1}}}
\newcommand{\DataTypeTok}[1]{\textcolor[rgb]{0.13,0.29,0.53}{#1}}
\newcommand{\DecValTok}[1]{\textcolor[rgb]{0.00,0.00,0.81}{#1}}
\newcommand{\DocumentationTok}[1]{\textcolor[rgb]{0.56,0.35,0.01}{\textbf{\textit{#1}}}}
\newcommand{\ErrorTok}[1]{\textcolor[rgb]{0.64,0.00,0.00}{\textbf{#1}}}
\newcommand{\ExtensionTok}[1]{#1}
\newcommand{\FloatTok}[1]{\textcolor[rgb]{0.00,0.00,0.81}{#1}}
\newcommand{\FunctionTok}[1]{\textcolor[rgb]{0.00,0.00,0.00}{#1}}
\newcommand{\ImportTok}[1]{#1}
\newcommand{\InformationTok}[1]{\textcolor[rgb]{0.56,0.35,0.01}{\textbf{\textit{#1}}}}
\newcommand{\KeywordTok}[1]{\textcolor[rgb]{0.13,0.29,0.53}{\textbf{#1}}}
\newcommand{\NormalTok}[1]{#1}
\newcommand{\OperatorTok}[1]{\textcolor[rgb]{0.81,0.36,0.00}{\textbf{#1}}}
\newcommand{\OtherTok}[1]{\textcolor[rgb]{0.56,0.35,0.01}{#1}}
\newcommand{\PreprocessorTok}[1]{\textcolor[rgb]{0.56,0.35,0.01}{\textit{#1}}}
\newcommand{\RegionMarkerTok}[1]{#1}
\newcommand{\SpecialCharTok}[1]{\textcolor[rgb]{0.00,0.00,0.00}{#1}}
\newcommand{\SpecialStringTok}[1]{\textcolor[rgb]{0.31,0.60,0.02}{#1}}
\newcommand{\StringTok}[1]{\textcolor[rgb]{0.31,0.60,0.02}{#1}}
\newcommand{\VariableTok}[1]{\textcolor[rgb]{0.00,0.00,0.00}{#1}}
\newcommand{\VerbatimStringTok}[1]{\textcolor[rgb]{0.31,0.60,0.02}{#1}}
\newcommand{\WarningTok}[1]{\textcolor[rgb]{0.56,0.35,0.01}{\textbf{\textit{#1}}}}
\usepackage{graphicx}
% grffile has become a legacy package: https://ctan.org/pkg/grffile
\IfFileExists{grffile.sty}{%
\usepackage{grffile}
}{}
\makeatletter
\def\maxwidth{\ifdim\Gin@nat@width>\linewidth\linewidth\else\Gin@nat@width\fi}
\def\maxheight{\ifdim\Gin@nat@height>\textheight\textheight\else\Gin@nat@height\fi}
\makeatother
% Scale images if necessary, so that they will not overflow the page
% margins by default, and it is still possible to overwrite the defaults
% using explicit options in \includegraphics[width, height, ...]{}
\setkeys{Gin}{width=\maxwidth,height=\maxheight,keepaspectratio}
\IfFileExists{parskip.sty}{%
\usepackage{parskip}
}{% else
\setlength{\parindent}{0pt}
\setlength{\parskip}{6pt plus 2pt minus 1pt}
}
\setlength{\emergencystretch}{3em}  % prevent overfull lines
\providecommand{\tightlist}{%
  \setlength{\itemsep}{0pt}\setlength{\parskip}{0pt}}
\setcounter{secnumdepth}{0}
% Redefines (sub)paragraphs to behave more like sections
\ifx\paragraph\undefined\else
\let\oldparagraph\paragraph
\renewcommand{\paragraph}[1]{\oldparagraph{#1}\mbox{}}
\fi
\ifx\subparagraph\undefined\else
\let\oldsubparagraph\subparagraph
\renewcommand{\subparagraph}[1]{\oldsubparagraph{#1}\mbox{}}
\fi

%%% Use protect on footnotes to avoid problems with footnotes in titles
\let\rmarkdownfootnote\footnote%
\def\footnote{\protect\rmarkdownfootnote}

%%% Change title format to be more compact
\usepackage{titling}

% Create subtitle command for use in maketitle
\providecommand{\subtitle}[1]{
  \posttitle{
    \begin{center}\large#1\end{center}
    }
}

\setlength{\droptitle}{-2em}

  \title{}
    \pretitle{\vspace{\droptitle}}
  \posttitle{}
    \author{}
    \preauthor{}\postauthor{}
    \date{}
    \predate{}\postdate{}
  

\begin{document}

\#\emph{TEA TIME}

\#Load the tea dataset from the package Factominer. Explore the data
briefly: look at the structure and the dimensions of the data and
visualize it. Then do Multiple Correspondence Analysis on the tea data
(or to a certain columns of the data, it's up to you). Interpret the
results of the MCA and draw at least the variable biplot of the
analysis. You can also explore other plotting options for MCA. Comment
on the output of the plots. (0-4 points)

\begin{Shaded}
\begin{Highlighting}[]
\KeywordTok{library}\NormalTok{(FactoMineR)}
\KeywordTok{library}\NormalTok{(ggplot2)}
\KeywordTok{library}\NormalTok{(dplyr)}
\end{Highlighting}
\end{Shaded}

\begin{verbatim}
## 
## Attaching package: 'dplyr'
\end{verbatim}

\begin{verbatim}
## The following objects are masked from 'package:stats':
## 
##     filter, lag
\end{verbatim}

\begin{verbatim}
## The following objects are masked from 'package:base':
## 
##     intersect, setdiff, setequal, union
\end{verbatim}

\begin{Shaded}
\begin{Highlighting}[]
\KeywordTok{library}\NormalTok{(tidyr)}
\KeywordTok{data}\NormalTok{(}\StringTok{"tea"}\NormalTok{)}
\KeywordTok{str}\NormalTok{(tea)}
\end{Highlighting}
\end{Shaded}

\begin{verbatim}
## 'data.frame':    300 obs. of  36 variables:
##  $ breakfast       : Factor w/ 2 levels "breakfast","Not.breakfast": 1 1 2 2 1 2 1 2 1 1 ...
##  $ tea.time        : Factor w/ 2 levels "Not.tea time",..: 1 1 2 1 1 1 2 2 2 1 ...
##  $ evening         : Factor w/ 2 levels "evening","Not.evening": 2 2 1 2 1 2 2 1 2 1 ...
##  $ lunch           : Factor w/ 2 levels "lunch","Not.lunch": 2 2 2 2 2 2 2 2 2 2 ...
##  $ dinner          : Factor w/ 2 levels "dinner","Not.dinner": 2 2 1 1 2 1 2 2 2 2 ...
##  $ always          : Factor w/ 2 levels "always","Not.always": 2 2 2 2 1 2 2 2 2 2 ...
##  $ home            : Factor w/ 2 levels "home","Not.home": 1 1 1 1 1 1 1 1 1 1 ...
##  $ work            : Factor w/ 2 levels "Not.work","work": 1 1 2 1 1 1 1 1 1 1 ...
##  $ tearoom         : Factor w/ 2 levels "Not.tearoom",..: 1 1 1 1 1 1 1 1 1 2 ...
##  $ friends         : Factor w/ 2 levels "friends","Not.friends": 2 2 1 2 2 2 1 2 2 2 ...
##  $ resto           : Factor w/ 2 levels "Not.resto","resto": 1 1 2 1 1 1 1 1 1 1 ...
##  $ pub             : Factor w/ 2 levels "Not.pub","pub": 1 1 1 1 1 1 1 1 1 1 ...
##  $ Tea             : Factor w/ 3 levels "black","Earl Grey",..: 1 1 2 2 2 2 2 1 2 1 ...
##  $ How             : Factor w/ 4 levels "alone","lemon",..: 1 3 1 1 1 1 1 3 3 1 ...
##  $ sugar           : Factor w/ 2 levels "No.sugar","sugar": 2 1 1 2 1 1 1 1 1 1 ...
##  $ how             : Factor w/ 3 levels "tea bag","tea bag+unpackaged",..: 1 1 1 1 1 1 1 1 2 2 ...
##  $ where           : Factor w/ 3 levels "chain store",..: 1 1 1 1 1 1 1 1 2 2 ...
##  $ price           : Factor w/ 6 levels "p_branded","p_cheap",..: 4 6 6 6 6 3 6 6 5 5 ...
##  $ age             : int  39 45 47 23 48 21 37 36 40 37 ...
##  $ sex             : Factor w/ 2 levels "F","M": 2 1 1 2 2 2 2 1 2 2 ...
##  $ SPC             : Factor w/ 7 levels "employee","middle",..: 2 2 4 6 1 6 5 2 5 5 ...
##  $ Sport           : Factor w/ 2 levels "Not.sportsman",..: 2 2 2 1 2 2 2 2 2 1 ...
##  $ age_Q           : Factor w/ 5 levels "15-24","25-34",..: 3 4 4 1 4 1 3 3 3 3 ...
##  $ frequency       : Factor w/ 4 levels "1/day","1 to 2/week",..: 1 1 3 1 3 1 4 2 3 3 ...
##  $ escape.exoticism: Factor w/ 2 levels "escape-exoticism",..: 2 1 2 1 1 2 2 2 2 2 ...
##  $ spirituality    : Factor w/ 2 levels "Not.spirituality",..: 1 1 1 2 2 1 1 1 1 1 ...
##  $ healthy         : Factor w/ 2 levels "healthy","Not.healthy": 1 1 1 1 2 1 1 1 2 1 ...
##  $ diuretic        : Factor w/ 2 levels "diuretic","Not.diuretic": 2 1 1 2 1 2 2 2 2 1 ...
##  $ friendliness    : Factor w/ 2 levels "friendliness",..: 2 2 1 2 1 2 2 1 2 1 ...
##  $ iron.absorption : Factor w/ 2 levels "iron absorption",..: 2 2 2 2 2 2 2 2 2 2 ...
##  $ feminine        : Factor w/ 2 levels "feminine","Not.feminine": 2 2 2 2 2 2 2 1 2 2 ...
##  $ sophisticated   : Factor w/ 2 levels "Not.sophisticated",..: 1 1 1 2 1 1 1 2 2 1 ...
##  $ slimming        : Factor w/ 2 levels "No.slimming",..: 1 1 1 1 1 1 1 1 1 1 ...
##  $ exciting        : Factor w/ 2 levels "exciting","No.exciting": 2 1 2 2 2 2 2 2 2 2 ...
##  $ relaxing        : Factor w/ 2 levels "No.relaxing",..: 1 1 2 2 2 2 2 2 2 2 ...
##  $ effect.on.health: Factor w/ 2 levels "effect on health",..: 2 2 2 2 2 2 2 2 2 2 ...
\end{verbatim}

\begin{Shaded}
\begin{Highlighting}[]
\KeywordTok{dim}\NormalTok{(tea)}
\end{Highlighting}
\end{Shaded}

\begin{verbatim}
## [1] 300  36
\end{verbatim}

\begin{Shaded}
\begin{Highlighting}[]
\NormalTok{keep_columns <-}\StringTok{ }\KeywordTok{c}\NormalTok{(}\StringTok{"Tea"}\NormalTok{, }\StringTok{"How"}\NormalTok{, }\StringTok{"how"}\NormalTok{, }\StringTok{"sugar"}\NormalTok{, }\StringTok{"where"}\NormalTok{, }\StringTok{"lunch"}\NormalTok{)}
\NormalTok{tea_time <-}\StringTok{ }\KeywordTok{select}\NormalTok{(tea, keep_columns)}
\KeywordTok{summary}\NormalTok{(tea_time)}
\end{Highlighting}
\end{Shaded}

\begin{verbatim}
##         Tea         How                      how           sugar    
##  black    : 74   alone:195   tea bag           :170   No.sugar:155  
##  Earl Grey:193   lemon: 33   tea bag+unpackaged: 94   sugar   :145  
##  green    : 33   milk : 63   unpackaged        : 36                 
##                  other:  9                                          
##                   where           lunch    
##  chain store         :192   lunch    : 44  
##  chain store+tea shop: 78   Not.lunch:256  
##  tea shop            : 30                  
## 
\end{verbatim}

\begin{Shaded}
\begin{Highlighting}[]
\KeywordTok{str}\NormalTok{(tea_time)}
\end{Highlighting}
\end{Shaded}

\begin{verbatim}
## 'data.frame':    300 obs. of  6 variables:
##  $ Tea  : Factor w/ 3 levels "black","Earl Grey",..: 1 1 2 2 2 2 2 1 2 1 ...
##  $ How  : Factor w/ 4 levels "alone","lemon",..: 1 3 1 1 1 1 1 3 3 1 ...
##  $ how  : Factor w/ 3 levels "tea bag","tea bag+unpackaged",..: 1 1 1 1 1 1 1 1 2 2 ...
##  $ sugar: Factor w/ 2 levels "No.sugar","sugar": 2 1 1 2 1 1 1 1 1 1 ...
##  $ where: Factor w/ 3 levels "chain store",..: 1 1 1 1 1 1 1 1 2 2 ...
##  $ lunch: Factor w/ 2 levels "lunch","Not.lunch": 2 2 2 2 2 2 2 2 2 2 ...
\end{verbatim}

\begin{Shaded}
\begin{Highlighting}[]
\KeywordTok{gather}\NormalTok{(tea_time) }\OperatorTok\StringTok{ }\KeywordTok{ggplot}\NormalTok{(}\KeywordTok{aes}\NormalTok{(value)) }\OperatorTok{+}\StringTok{ }\KeywordTok{facet_wrap}\NormalTok{(}\StringTok{"key"}\NormalTok{, }\DataTypeTok{scales =} \StringTok{"free"}\NormalTok{) }\OperatorTok{+}\StringTok{ }\KeywordTok{geom_bar}\NormalTok{() }\OperatorTok{+}\StringTok{ }\KeywordTok{theme}\NormalTok{(}\DataTypeTok{axis.text.x =} \KeywordTok{element_text}\NormalTok{(}\DataTypeTok{angle =} \DecValTok{45}\NormalTok{, }\DataTypeTok{hjust =} \DecValTok{1}\NormalTok{, }\DataTypeTok{size =} \DecValTok{8}\NormalTok{))}
\end{Highlighting}
\end{Shaded}

\begin{verbatim}
## Warning: attributes are not identical across measure variables;
## they will be dropped
\end{verbatim}

\includegraphics{chapter5_files/figure-latex/unnamed-chunk-1-1.pdf}

\begin{Shaded}
\begin{Highlighting}[]
\NormalTok{mca <-}\StringTok{ }\KeywordTok{MCA}\NormalTok{(tea_time, }\DataTypeTok{graph =} \OtherTok{FALSE}\NormalTok{)}
\KeywordTok{summary}\NormalTok{(mca)}
\end{Highlighting}
\end{Shaded}

\begin{verbatim}
## 
## Call:
## MCA(X = tea_time, graph = FALSE) 
## 
## 
## Eigenvalues
##                        Dim.1   Dim.2   Dim.3   Dim.4   Dim.5   Dim.6   Dim.7
## Variance               0.279   0.261   0.219   0.189   0.177   0.156   0.144
## % of var.             15.238  14.232  11.964  10.333   9.667   8.519   7.841
## Cumulative % of var.  15.238  29.471  41.435  51.768  61.434  69.953  77.794
##                        Dim.8   Dim.9  Dim.10  Dim.11
## Variance               0.141   0.117   0.087   0.062
## % of var.              7.705   6.392   4.724   3.385
## Cumulative % of var.  85.500  91.891  96.615 100.000
## 
## Individuals (the 10 first)
##                       Dim.1    ctr   cos2    Dim.2    ctr   cos2    Dim.3
## 1                  | -0.298  0.106  0.086 | -0.328  0.137  0.105 | -0.327
## 2                  | -0.237  0.067  0.036 | -0.136  0.024  0.012 | -0.695
## 3                  | -0.369  0.162  0.231 | -0.300  0.115  0.153 | -0.202
## 4                  | -0.530  0.335  0.460 | -0.318  0.129  0.166 |  0.211
## 5                  | -0.369  0.162  0.231 | -0.300  0.115  0.153 | -0.202
## 6                  | -0.369  0.162  0.231 | -0.300  0.115  0.153 | -0.202
## 7                  | -0.369  0.162  0.231 | -0.300  0.115  0.153 | -0.202
## 8                  | -0.237  0.067  0.036 | -0.136  0.024  0.012 | -0.695
## 9                  |  0.143  0.024  0.012 |  0.871  0.969  0.435 | -0.067
## 10                 |  0.476  0.271  0.140 |  0.687  0.604  0.291 | -0.650
##                       ctr   cos2  
## 1                   0.163  0.104 |
## 2                   0.735  0.314 |
## 3                   0.062  0.069 |
## 4                   0.068  0.073 |
## 5                   0.062  0.069 |
## 6                   0.062  0.069 |
## 7                   0.062  0.069 |
## 8                   0.735  0.314 |
## 9                   0.007  0.003 |
## 10                  0.643  0.261 |
## 
## Categories (the 10 first)
##                        Dim.1     ctr    cos2  v.test     Dim.2     ctr    cos2
## black              |   0.473   3.288   0.073   4.677 |   0.094   0.139   0.003
## Earl Grey          |  -0.264   2.680   0.126  -6.137 |   0.123   0.626   0.027
## green              |   0.486   1.547   0.029   2.952 |  -0.933   6.111   0.107
## alone              |  -0.018   0.012   0.001  -0.418 |  -0.262   2.841   0.127
## lemon              |   0.669   2.938   0.055   4.068 |   0.531   1.979   0.035
## milk               |  -0.337   1.420   0.030  -3.002 |   0.272   0.990   0.020
## other              |   0.288   0.148   0.003   0.876 |   1.820   6.347   0.102
## tea bag            |  -0.608  12.499   0.483 -12.023 |  -0.351   4.459   0.161
## tea bag+unpackaged |   0.350   2.289   0.056   4.088 |   1.024  20.968   0.478
## unpackaged         |   1.958  27.432   0.523  12.499 |  -1.015   7.898   0.141
##                     v.test     Dim.3     ctr    cos2  v.test  
## black                0.929 |  -1.081  21.888   0.382 -10.692 |
## Earl Grey            2.867 |   0.433   9.160   0.338  10.053 |
## green               -5.669 |  -0.108   0.098   0.001  -0.659 |
## alone               -6.164 |  -0.113   0.627   0.024  -2.655 |
## lemon                3.226 |   1.329  14.771   0.218   8.081 |
## milk                 2.422 |   0.013   0.003   0.000   0.116 |
## other                5.534 |  -2.524  14.526   0.197  -7.676 |
## tea bag             -6.941 |  -0.065   0.183   0.006  -1.287 |
## tea bag+unpackaged  11.956 |   0.019   0.009   0.000   0.226 |
## unpackaged          -6.482 |   0.257   0.602   0.009   1.640 |
## 
## Categorical variables (eta2)
##                      Dim.1 Dim.2 Dim.3  
## Tea                | 0.126 0.108 0.410 |
## How                | 0.076 0.190 0.394 |
## how                | 0.708 0.522 0.010 |
## sugar              | 0.065 0.001 0.336 |
## where              | 0.702 0.681 0.055 |
## lunch              | 0.000 0.064 0.111 |
\end{verbatim}

\begin{Shaded}
\begin{Highlighting}[]
\KeywordTok{plot}\NormalTok{(mca, }\DataTypeTok{invisible=}\KeywordTok{c}\NormalTok{(}\StringTok{"ind"}\NormalTok{), }\DataTypeTok{habillage =} \StringTok{"quali"}\NormalTok{)}
\end{Highlighting}
\end{Shaded}

\includegraphics{chapter5_files/figure-latex/unnamed-chunk-2-1.pdf}

\hypertarget{the-mca-is-a-very-useful-tool-to-analyze-a-non-numerical-nominal-cathegorcal-qualitative-data.-among-others-it-provides-insights-into-existing-patterns-in-the-data.-from-the-example-of-the-mca-abovethat-explors-tea-drinking-habits-we-can-get-insights-into-various-aspects-of-the-data-at-once.-also-some-interesting-relationships-are-revealed.-for-example-we-can-see-that-earl-grey-is-more-likely-to-be-drunk-with-milk-than-lemon-or-alone.-its-also-more-likely-to-be-drunk-with-sugar-than-without.-the-black-tea-on-the-other-hand-is-more-likely-to-be-taken-without-sugar-as-well-as-more-likely-to-be-enjoyed-with-lemon-than-with-milk.-most-individuals-dont-drink-tea-with-lunch.}{%
\section{The MCA is a very useful tool to analyze a non-numerical,
nominal cathegorcal, qualitative data. Among others, it provides
insights into existing patterns in the data. From the example of the MCA
above,that explors tea-drinking habits, we can get insights into various
aspects of the data at once. Also, some interesting relationships are
revealed. For example, we can see that Earl Grey is more likely to be
drunk with milk, than lemon or alone. It's also more likely to be drunk
with sugar than without. The black tea, on the other hand, is more
likely to be taken without sugar, as well as more likely to be enjoyed
with lemon than with milk. Most individuals don't drink tea with
lunch.}\label{the-mca-is-a-very-useful-tool-to-analyze-a-non-numerical-nominal-cathegorcal-qualitative-data.-among-others-it-provides-insights-into-existing-patterns-in-the-data.-from-the-example-of-the-mca-abovethat-explors-tea-drinking-habits-we-can-get-insights-into-various-aspects-of-the-data-at-once.-also-some-interesting-relationships-are-revealed.-for-example-we-can-see-that-earl-grey-is-more-likely-to-be-drunk-with-milk-than-lemon-or-alone.-its-also-more-likely-to-be-drunk-with-sugar-than-without.-the-black-tea-on-the-other-hand-is-more-likely-to-be-taken-without-sugar-as-well-as-more-likely-to-be-enjoyed-with-lemon-than-with-milk.-most-individuals-dont-drink-tea-with-lunch.}}

\begin{Shaded}
\begin{Highlighting}[]
\KeywordTok{plot}\NormalTok{(mca, }\DataTypeTok{invisible=}\KeywordTok{c}\NormalTok{(}\StringTok{"var"}\NormalTok{), }\DataTypeTok{habillage =} \StringTok{"quali"}\NormalTok{)}
\end{Highlighting}
\end{Shaded}

\includegraphics{chapter5_files/figure-latex/unnamed-chunk-3-1.pdf} \#
The MCA above shows exclusivly the distribution of data concerning
individuals, suggesting similarities and dissimilarities among them.
They are arranged in clear clusters.

\begin{Shaded}
\begin{Highlighting}[]
\KeywordTok{library}\NormalTok{(factoextra)}
\end{Highlighting}
\end{Shaded}

\begin{verbatim}
## Welcome! Related Books: `Practical Guide To Cluster Analysis in R` at https://goo.gl/13EFCZ
\end{verbatim}

\begin{Shaded}
\begin{Highlighting}[]
\KeywordTok{data}\NormalTok{(}\StringTok{"tea_time"}\NormalTok{)}
\end{Highlighting}
\end{Shaded}

\begin{verbatim}
## Warning in data("tea_time"): data set 'tea_time' not found
\end{verbatim}

\begin{Shaded}
\begin{Highlighting}[]
\NormalTok{res.mca <-}\StringTok{ }\KeywordTok{MCA}\NormalTok{(tea_time, }\DataTypeTok{graph=}\OtherTok{FALSE}\NormalTok{)}
\KeywordTok{fviz_mca_biplot}\NormalTok{(res.mca, }\DataTypeTok{repel =} \OtherTok{TRUE}\NormalTok{, }\DataTypeTok{ggtheme =} \KeywordTok{theme_minimal}\NormalTok{())}
\end{Highlighting}
\end{Shaded}

\includegraphics{chapter5_files/figure-latex/unnamed-chunk-4-1.pdf}

\hypertarget{this-extensive-plot-shows-both-variables-and-individuals-at-the-same-time-highliting-relationships-among-them.-the-distance-measures-the-similarity-and-dissimilarity-among-the-variables-and-individuas-that-is-the-possible-correlations-between-them.}{%
\section{This extensive plot shows both, variables and individuals at
the same time, highliting relationships among them. The distance
measures the similarity and dissimilarity among the variables and
individuas, that is the possible correlations between
them.}\label{this-extensive-plot-shows-both-variables-and-individuals-at-the-same-time-highliting-relationships-among-them.-the-distance-measures-the-similarity-and-dissimilarity-among-the-variables-and-individuas-that-is-the-possible-correlations-between-them.}}


\end{document}
